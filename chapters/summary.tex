In the course of the work various approach to data augmentation were explored.
Three different deep leaning architectures were introduced and compared with the traditional approach.
The augmentation networks were trained on real--life data and used to export new datasets to fit the super--resolution algorithm.
Then the super--resolution models were evaluated on various synthetic and real--file images; both in numerical and visual way.
The cross--validation based approach to training supervision proved to be valuable.
The deep learning based methods proved to be of utility, yet slightly below what would be expected, since they didn't surpass the bicubic interpolation approach.
There may be a variety of reasons for that to explore.
Many possibilities were not explored in a full way; translations between low--resolution images, may be investigated in a greater detail.
Modeling them after distribution extracted from a real--world dataset may lead to enhanced results.
The semi--simulated approach for data generation was not included in the scope of the work, however it would be interesting to investigate this option in the future.

The most novel observation in the work is connected with the cross--validation technique during training.
Early stopping training based on this approach lead to good results.
This indicates a greater overfitting problem in regard to data generation method.
The robust data augmentation problem remains open, however valuable observations about issue of super--resolution generalization has been made.

There is value in the opportunity for further development of the work and more detailed investigation.
Achieving robustness and greater generalization of super--resolution techniques through better data augmentation is an important step towards productization of this technique.
A super--resolution algorithm independent of satellite data type would be useful in image  processing pipeline during aerospace missions, remote sensing tasks and Earth observations.