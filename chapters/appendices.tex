\chapter{Model and training parameters of the augmentation networks}
\label{ch:appendix-params}
As stated in the work the models and training process are parametrized by set o variables in a \textit{params.yaml} file.
This file is supervised by Git and \gls{dvc} to ensure reproducibility and artifacts caching.
Contents of the configuration file during the training of augmentation networks are presented in Listing \ref{lst:params-file}.
\begin{longlisting}
\inputminted{yaml}{listings/params.yaml}
\caption{Contents of the \textit{params.yaml} file used for training}
\label{lst:params-file}
\end{longlisting}

\chapter{Technical documentation}
\label{ch:appendix-technical}
As mentioned in Section \ref{sec:exp-management} trainings are done either done with the aid of the \gls{dvc} system or by running scripts manually.
The trainings done automatically by the \gls{dvc} system contain the word \textit{dvc} as the experiment name.
The status of these trainings is supervised by \gls{dvc}, current progress can be checked with the \mintinline[breaklines]{shell}{dvc status} command.
Trainings are automatically rerun or recached based on configuration files after running the \mintinline[breaklines]{shell}{dvc repro} instruction.
Manual trainings are run by invoking Python directly.
This can be done with the command: \mintinline[breaklines]{shell}{python -m cnn_res_degrader.train [-s] [-a] [-g] training_name} where the optional arguments decide which architecture to train and the positional argument is the experiment name.

Validation on Proba-V dataset can be run with the command: \mintinline[breaklines]{shell}{python -m cnn_res_degrader.test [-s|-a|-g] weights_path output_dir} where the positional stores model architecture and positional arguments hold paths to trained model weights and output directory.

The augmented Sentinel-2 datasets can be created by running the command: \mintinline[breaklines]{shell}{python -m export_sentinel.py [-s|-a|-g] [-r] [-d] weights_path}, where architecture and path to weights are denoted as described earlier.
The last two of the switches enable generating random translations (instead of using ones saved in sentinel files) and running a demo export on one image (instead of using the whole dataset).

\chapter{Supplementary media}
The supplementary media include:
\begin{itemize}
	\item This document.
	\item Source code files.
	\item The datasets can be fetched from the cloud using \gls{dvc}; to gain access to the data contact the author for the passcodes.
\end{itemize}
