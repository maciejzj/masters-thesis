% Paper, encoding and fonts settings
\documentclass{report}
\usepackage{geometry}
\usepackage[utf8]{inputenc}
\usepackage[T1]{fontenc}

% Document structure and layout
\usepackage{pdfpages}
\usepackage[toc,page]{appendix}

% Graphics and plotting
\usepackage{graphicx}
\graphicspath{ {graphics/} }

% Math, physics and numbers
\usepackage{amsmath}
\usepackage{siunitx}

% Extra features and enhancements
\usepackage{cite}
% Tables
\usepackage{booktabs}
% Captions
\usepackage[justification=centering]{caption}
\usepackage{subcaption}
% Listings
\usepackage{minted}
% Make sure that hyperref is last loaded package
\usepackage[hidelinks]{hyperref}

% Define centered column type with fixed width
\newcolumntype{C}[1]{>{\centering\arraybackslash}p{#1}}
\begin{document}
\pagenumbering{gobble}

% Introductory pages
\begin{abstract}
The aim of this work is to enhance super--resolution satellite imaging by using
data augmentation techniques based on deep learning algorithms.
Super--resolution is a technology that enables upscaling images to a higher
resolution with more refined details and improved quality.
Such image--enhancing techniques are nowadays undergoing rapid development
thanks to advancements in deep learning and convolutional neural networks.
Deep learning is an approach in which training data plays a key role in the
outcome and quality of the solution.
Size and quality of the dataset used to train super--resolution networks are
crucial to achieve best results.
This is especially significant when working with satellite images, which are
effortful to acquire in large numbers.
Thus, when training a super--resolution network, it may be worth incorporating
data augmentation techniques.
Data augmentation is a process that intends to enlarge and improve training
datasets for machine learning by transforming, multiplying or generating data.
This process has been traditionally done using trivial techniques, however this
work aims to use deep learning to generate datasets for training
super--resolution algorithms.
Following chapters provide an overview of modern super--resolution solutions
and a proposal of deep learning algorithms to enhance the training datasets.
Results of the work are evaluated by testing super--resolution networks which
were trained on the datasets created during the project.
\end{abstract}

\tableofcontents

\pagenumbering{arabic}

% Chapters
\chapter{Introduction}
\section{Super-resolution technology}
Super-resolution is a~group of techniques that upscale images and improve
their quality.
A super-resolution algorithm can be treated as a function that takes an image and returns
it with a resolution \textit{n} times larger \cite{wang-2019-srsurvey}.
Algorithms that were taken into account in the process usually upscale images two or four times.
In this study, due to the size of images in the available datasets (Proba-V \cite{esa-proba} and Sentinel-2 \cite{esa-sentinel}, more on the utilized data can be found in Chapter \ref{ch:scope}), a ratio of three was used.

It is important to distinguish between super-resolution and traditional
upscaling algorithms.
The latter use interpolation to enlarge pictures; however, they hardly improve the image quality.
The intent of super-resolution is not only to upscale images but to improve
the quality and details.
Nowadays, such an effect is achieved using machine learning, precisely---deep
learning---a technology that utilizes multi-layered neural networks trained
with large datasets.
Deep learning networks that process image data usually include convolutional
layers.
Such layers contain a number of image filters whose weights are tuned during the training process.
Like all the rest of the machine learning algorithms, the deep learning-based
super-resolution works in a statistical manner.
This means that the extra details created during the image enhancement process
state an imagined approximation of image features.
It is important to keep the statistical nature of machine learning algorithms
in mind.

Two main kinds of super-resolution algorithms can be outlined:
\textit{one-to-one} and \textit{many-to-one}.
The first one is the obvious approach where a single low-resolution image is
translated into a high-resolution picture.
The second method is to combine multiple low-resolution images into a single high-resolution one.
The first technique is often called \textit{single-image super-resolution} and the latter \textit{multi-image super-resolution}.

Super-resolution imaging is a technique relevant in the fields of satellite imaging,
remote sensing, and geoscience.
This work focuses on the application of super-resolution technology to this type of data.
The most common cause for image enhancement is aesthetics.
This application is viable in satellite imaging; however, super-resolution can
lead to other practical advantages.
Image enhancing techniques can be used as a preprocessing step in remote
sensing pipelines.
For this reason, super-resolution can be especially useful when considered in
the context of satellite imagery.
Multi-image super-resolution in satellite applications can be treated as a data fusion operation.
A visual demonstration of modern super-resolution applied on satellite images can be seen in Figure \ref{fig:super-res-demo}.
In these examples, a set of nine low-resolution images per scene was super-resolved into a single high-resolution one.
A sample of those nine images is shown in the comparison.
\begin{figure}
    \centering
    \begin{subfigure}[t]{0.32\textwidth}
        \centering
        \includegraphics[width=\textwidth]{high_res_net_demo_fields_lr}
        \caption{One of the nine low-resolution images of a scene}
        \label{fig:super-res-demo-lra}
    \end{subfigure}
    \hfill
    \begin{subfigure}[t]{0.32\textwidth}
        \centering
        \includegraphics[width=\textwidth]{high_res_net_demo_fields_sr}
        \caption{Super-resolution reconstruction of the scene}
        \label{fig:super-res-demo-sra}
    \end{subfigure}
    \hfill
    \begin{subfigure}[t]{0.32\textwidth}
        \centering
        \includegraphics[width=\textwidth]{high_res_net_demo_fields_hr}
        \caption{Real-life high-resolution picture of the scene}
        \label{fig:super-res-demo-hra}
    \end{subfigure}
    \centering
    \vskip\baselineskip
    \begin{subfigure}[t]{0.32\textwidth}
        \centering
        \includegraphics[width=\textwidth]{high_res_net_demo_dots_lr}
        \caption{One of the nine low-resolution images of a scene}
        \label{fig:super-res-demo-lrb}
    \end{subfigure}
    \hfill
    \begin{subfigure}[t]{0.32\textwidth}
        \centering
        \includegraphics[width=\textwidth]{high_res_net_demo_dots_sr}
        \caption{Super-resolution reconstruction of the scene}
        \label{fig:super-res-demo-srb}
    \end{subfigure}
    \hfill
    \begin{subfigure}[t]{0.32\textwidth}
        \centering
        \includegraphics[width=\textwidth]{high_res_net_demo_dots_hr}
        \caption{Real-life high-resolution picture of the scene}
        \label{fig:super-res-demo-hrb}
    \end{subfigure}
    \captionsetup{justification=justified}
    \caption{A demonstration of super-resolution technique, where nine low-resolution satellite images of different landscapes are turned into one  high-resolution image; \ref{fig:super-res-demo-lra}, \ref{fig:super-res-demo-sra}, and \ref{fig:super-res-demo-hra} show pictures of a rural area; \ref{fig:super-res-demo-lrb}, \ref{fig:super-res-demo-srb}, and \ref{fig:super-res-demo-hrb} present pictures of a barren landscape with silos (these examples comes from the HighRes-net model applied on the Proba-V dataset) \cite{deudon-2020-highresnet}}
    \label{fig:super-res-demo}
\end{figure}

Deep-learning based image-enhancing algorithms may be domain specific.
Different model architectures and data are used to create super-resolution algorithms for satellite imagery and human faces.

\section{Purpose of data augmentation and available solutions}
\label{sec:augmentation-introduction}
\textit{Data augmentation} is a suite of techniques that enhance the size and quality of datasets used for training machine learning models \cite{perez-2017-augmentation} \cite{mikolajczyk-2018-augmentation}.
Deep learning, utilized in the modern super-resolution solutions, requires a
lot of data to train successfully.
An increase in the quality and size of a dataset can lead to far better results when
training a neural network.
This is why data augmentation techniques are often used to improve the performance of deep networks.
Data augmentation incorporates various transformations to improve, multiply, or
generate training data.
This process often multiplies data by modifying existing examples; however, more advanced generation techniques can produce fully artificial datasets.
Completely new data created in the augmentation process is often called \textit{synthetic} \cite{nikolenko-2021-synthetic}.
Classical augmentation techniques utilize simple operations like image resizing, flipping, mirroring, recoloring, etc.
However, a different approach can be taken to generate data.
It is possible to train deep neural networks to perform data augmentation
transformations.
This approach can be especially useful when the available dataset is too small to train the desired network.
The small dataset can be used to fit a simpler network that is taught to create new data, based on the existing training set.
Then, this small augmentation network can be used to generate more data for the
original model to learn.
With deep learning capabilities, networks can learn to multiply, transform, or
even generate data without direct input.

Generating more diverse training datasets is often used as a method of \textit{regularization}.
Regularization methods prevent overfitting machine learning models to the training datasets which leads to better generalization capabilities \cite{kukacka2017regularization}.
The problem of overfitting and the role of data augmentation in preventing it are covered in more detail in Sections \ref{sec:neural-nets} and \ref{sec:data-augmentation-in-ml}.

\section{Aim of the work and motivation}
The objective of this work is to create a set of augmentation networks for
enhancing super-resolution training data.
Subsequent chapters will present the considered super-resolution architectures
with more details and propose neural network models for data augmentation.
The task of the augmentation algorithms is to create low-resolution images from high-resolution ones, to make training pairs for super-resolution networks. 

The nature of super-resolution technology and satellite imagery imposes certain
ways in which augmentation should be applied to the training data.
Single image super-resolution is trained using pairs of high and low-resolution images.
On the other hand, multi-image super-resolution learns from examples of one high-resolution image bounded with multiple low-resolution ones.
These requirements render compiling training sets a challenge, especially in the
field of satellite imagery.
Even though single satellite images are widely available, it is difficult to find collections containing pairs of the same scenes with different resolutions.

Data augmentation and generation techniques may be the solution to the scarcity of satellite data suitable for super-resolution trainings.
Pairs of high and low-resolution images can be created from single-image satellite datasets by resizing existing photographs.
A set of training pairs can be created by downscaling high-resolution images.
In the case of many-to-one networks, a single high-resolution image can be
multiplied and shifted before shrinking to create more low-resolution images.
Such a technique may work well; however, it infuses the data with information
about resizing algorithms.
The way high-resolution and low-resolution images relate in such a set
depends on the interpolation algorithm (e.g., bicubic, bilinear,
nearest-neighbor, Lanczos).
The network trained on alike datasets will likely learn to invert given
interpolation methods.
This may not match exactly real-life scenarios where images are not created
using resizing algorithms.

However, another method of data augmentation based on deep learning is possible.
Similar to the super-resolution trainings, it requires pairs of real low and high-resolution
images of the same scene, taken by cameras of different quality.
The main idea of this project is to use a dataset of real-life data to
train an augmentation network.
Such a network would learn to create low-resolution images for
high-resolution ones, without imprinting resampling algorithms mechanisms into the
data.
The relation between low and high-resolution images in this augmented
dataset would resemble the relation between the same image taken by cameras of
different quality.
The deep learning approach to data augmentation has proven beneficial in other fields of super-resolution imaging \cite{bulat-2018-supergan}.

The augmentation neural network can be then used on other satellite image
datasets to generate training data pairs for super-resolution.
Different data, models, and generation techniques can be used to achieve desired
results.
Possible variations are discussed in the course of this work to improve
super-resolution datasets.
For clarity, the different methods of data augmentation are illustrated in Figures \ref{fig:augmentation-resize} and \ref{fig:augmentation-deep-learn}.
\begin{figure}
\begin{subfigure}{\textwidth}
    \centering
    \documentclass[tikz]{standalone}
\usepackage[utf8]{inputenc}

\usetikzlibrary{positioning}
\usetikzlibrary{shapes}
\usetikzlibrary{calc}

\begin{document}

\tikzstyle{cell} = [rectangle, rounded corners, minimum width=2cm, minimum height=1cm,text centered, draw=black, align=center]
\tikzset{arrow/.style={-stealth}}

\begin{tikzpicture}[ampersand replacement=\&]
    \node[cell, fill=blue!20] at (0,0) (S) {Single-resolution images\\(non--suitable for super-res training)};
    \node[cell] at (0, -2) (R) {Image resizing};
    \node[cell, fill=blue!20] at (0, -4) (P) {High and low--res pairs\\(suitable for super--res training)};
    
    \path[arrow]
    (S) edge node[right] {Augment} (R)
    (R) edge (P)
    ;
\end{tikzpicture}

\end{document}
    \caption{Super-resolution data augmentation with traditional resizing algorithms}
    \label{fig:augmentation-resize}
\end{subfigure}
\vskip\baselineskip
\begin{subfigure}{\textwidth}
    \centering
    \documentclass[tikz]{standalone}
\usepackage[utf8]{inputenc}

\usetikzlibrary{positioning}
\usetikzlibrary{shapes}
\usetikzlibrary{calc}

\begin{document}

\tikzstyle{cell} = [rectangle, rounded corners, minimum width=2cm, minimum height=1cm,text centered, draw=black, align=center]
\tikzset{arrow/.style={-stealth}}

\begin{tikzpicture}[ampersand replacement=\&]
    \node[cell, fill=blue!20] at (-8,-2) (SP) {Small dataset of\\high and low-resolution\\pairs};
    \node[cell, fill=blue!20] at (0,0) (S) {Single-resolution images\\(not suitable for super-resolution training\\bacause high and low-resolution pairs are missing)};
    \node[cell] at (0, -2) (A) {Augmentation network};
    \node[cell, fill=blue!20] at (0, -4) (P) {Low-resolution images\\(bundled with the original images\\create high and low-resolution pairs\\for super-resolution training)};
    
    \path[arrow]
    (SP) edge node[above]{Train} (A)
    (S) edge node[right]{Augment} (A)
    (A) edge (P)
    ;
\end{tikzpicture}

\end{document}
    \caption{Super-resolution data augmentation with deep-learning}
    \label{fig:augmentation-deep-learn}
\end{subfigure}
\end{figure}
Each of these starts with a dataset of single-resolution images.
To train a super-resolution algorithm a set of high and low-resolution pairs is needed.
Both of these methods create low-resolution counterparts for existing images to generate the training pairs.
The approach based on traditional resizing algorithms creates missing low-resolution pictures by downsizing existing ones.
However, with deep learning an existing set of high and low-resolution images is used to train the augmentation model.
This model learns to downsize existing images.
In a sense, the augmentation network learns to perform a task opposite to the super-resolution algorithm.
Then the augmentation model can be used to shrink existing images and create low-resolution images to be eventually used in the super-resolution training.

\section{Content outline}
The consecutive chapters provide a description of the theoretical background of regarded topics, implementation of data augmentation networks, and final evaluation.
The detailed content of the chapters is as follows:
\begin{description}
	\item [Chapter \ref{ch:introduction}] (this chapter) broadly introduces topics of super-resolution and data augmentation. The main goal of this work is set.
	\item [Chapter \ref{ch:analysis}] provides an overview of satellite imagery, explains topics of deep learning, super-resolution, augmentation, and image interpolation methods.
	\item [Chapter \ref{ch:scope}] explains the specific goals of the work. This chapter lays out the experiment plan and provides a rationale for chosen methods and tools.
	\item [Chapter \ref{ch:augmentation}] presents the main part of the work, contains data augmentation neural network architectures description.
	\item [Chapter \ref{ch:sr-evaluation}] includes the results of super-resolution training and evaluation of different datasets.
	\item [Chapter \ref{ch:summary}] summarizes the work and provides a commentary on the results.
\end{description}


\chapter{Overview of super--resolution imaging techniques}
\section{Characteristics of satellite imagery}
This works centers around super--resolution technique in the sphere of satellite imagery.
As mentioned in the introduction image enhancing can be domain specific.
This is especially crucial when satellite photos are taken into account.
Pictures taken from aerospace devices differ substantially from normal photography.
Mulit--image observation is usually favoured over single--image.
Satellites often take a series of photos of a single scene.
This puts emphasis on the mulit image super--resolution techniques in the many--to--one fashion.
Another unique feature of satellite observations is the usual spectral width of the imagery.
Scientific \textit{hyper--spectral} apparatus present on satellites, often take photos in a very wide spectrum that may not include frequencies of visible light.
This specific kind of image with large spectral dimension is often called a \textit{hyper--spectral cube}, because it can be represented as a three dimensional tensor (cube) with height, width and spectral dimensions.
Spectral bands in the cube can contain wavelengths such as infrared, near--infrared, panchromatic\footnote{A spectral range similar to the range of traditional monochromatic grayscale photography. This range is usually highlighted, because of connections with pre--digital imaging of the past century.}, radio frequencies and more.
Such images are often stored in special file formats or in a series of high bit--depth standard lossless image formats PNG or TIFF.
These can take up to 12 bands in different files per a single satellite photography.
Another crucial property of satellite imagery is the GSD (\textit{ground sample distance}) parameter, which denotes spacial distance between pixels of digital image.
For example, one--meter GSD states that location of adjacent pixels is one meter apart on the ground.

\section{Machine learning for image processing}
\subsection{Neural networks and deep learning}
\textit{Machine learning} is a computer science technique that solves problems by fitting algorithms to data, by optimization algorithms and statistics.
This approach contrasts with the traditional imperative problem--solving, where algorithms are designed with step--by--step attitude.
\textit{Artificial neural networks} are machine learning structures modeled after living organisms.
The traditional neural networks consist of layers of densely connected neurons.
Each of the neurons contains a set of inputs with connected weights.
The output of a neuron is passed through a nonlinear activation function.
The fitting process of such a network consists in adjusting the input weights.
This kind of machine learning architecture has been initially used with a set of predefined image filters.
A set of such filters would include basic geometric shapes.
These small filters would be convoluted with the input image.
Results of such an operation would be then fed into the neural network to get the final result of image processing.
With the advancements in the machine learning area a new kind of neural network layer was created---a \textit{convolutional layer}.
These layers consist of (one, two or even three dimensional) filters that cam be convoluted with the input image.
However, in  contrast to the older technique, these filters are adjusted in the fitting process of the network.
Elements in the filter tensor are treated like neuron weights and they are accommodated during gradient descent.
This enables creation of much better and flexible image processing neural networks.

However, the creation of convolutional neural networks leads to increasing complexity and number of parameters in models.
This issue can be addressed by using very large datasets for the fitting process.
Nowadays the smallest datasets for training modern neural networks contain thousands of images.
Such trainings require a lot of time and processing power; they usually must be performed using (even multiple) GPUs and may last a few days.
This combination of three factors: complex mulit--layered neural networks (often with media--oriented specialized layers), very large datasets (often with many classes and objects) and utilization of expensive time and resource--consuming trainings constitute what is called \textit{deep learning}.
This kind of machine learning has proven, in the last ten years, to hold a revolutionary potential, pushing forward techniques such as image and audio processing beyond what is possible with older methods.
Super--resolution, which this work revolves around, is possible thanks to advancements in the deep learning.

\subsection{Encoder--decoder mechanism}
Encoder--decoder network architecture is a common pattern in generative image processing.
It is used both in super--resolution models and in the data augmentation network presented in the latter chapters.
Encoder--decoder translates input data into abstract state during encoding, then reconstructs it when decoding.
The mid--point of the architecture usually bottlenecks the information containing compressed--like data.
Convolutional interpretation of the encoder--decoder is usually used when working with images.
During encoding process the depth of input is usually increased and spatial dimensions are shrunken.
This is achieved by subsequent usage of convolutional and pooling layers.
After encoding the compressed data can undergo some form of processing.
For example it can be flatten and dense connected, although this is rarely applied in the super--resolution, because dense layers break the fully convolutional nature of a network (meaning that it can not process images of varying spatial size).
The decoding process commonly reconstructs depth dimensions into spatial size by upsampling or transposed convolution.
The output may match input dimension, however it is not necessary.
In super--resolution it is common to output data of different size, than input.
Encoder--decoder architecture is appropriate for image--to--image transformations in machine learning.
The inner workings of such an architecture are shown in the figure \ref{fig:encoder-decoder}, where $ x $ and $ y $ denote input and output and $ z $ is the encoded hidden state. 
\begin{figure}
    \centering
    \documentclass[tikz]{standalone}
\usepackage[utf8]{inputenc}

\usetikzlibrary{positioning}
\usetikzlibrary{shapes.geometric}

\begin{document}
	
\tikzset{arrow/.style={-stealth}}

\begin{tikzpicture}
	\node[fill=blue!20, minimum width=0.5cm, minimum height=3.5cm] (X) at (0,0) {$\mathbf x$};
	
	\draw([xshift=0.5cm]X.north east) -- ([xshift=2.5cm,yshift=0.5cm]X.east) -- ([xshift=2.5cm,yshift=-0.5cm]X.east) -- ([xshift=0.5cm]X.south east) -- cycle; 
	\node at (1.75,0) {\textsc{Encoder}};
	
	\node[fill=blue!20, minimum width=0.5cm, minimum height=1.0cm] (Z) at (3.5cm,0) {$\mathbf z$};
	
	\draw([xshift=0.5cm]Z.north east) -- ([xshift=2.5cm,yshift=1.25cm]Z.north east) -- ([xshift=2.5cm,yshift=-1.25cm]Z.south east) -- ([xshift=0.5cm]Z.south east) -- cycle;
	\node at (5.25,0) {\textsc{Decoder}};
	
	\node[fill=blue!20, minimum width=0.5cm, minimum height=3.5cm] (Xp) at (7,0) {$\mathbf y$};
	
	\draw[arrow] (X.east) -- ([xshift=0.5cm]X.east);
	\draw[arrow] ([xshift=-0.5cm]Z.west) -- (Z.west);
	\draw[arrow] (Z.east) -- ([xshift=0.5cm]Z.east);
	\draw[arrow] ([xshift=-0.5cm]Xp.west) -- (Xp.west);
\end{tikzpicture}
\end{document}
    \caption{Schematic of encoder--decoder mechanism}
    \label{fig:encoder-decoder}
\end{figure}

Encoder--decoder mechanism is often enhanced with \textit{residual connections}.
These are often called \textit{skip connections}, because they form parallel branches in networks that skip certain operations.
These skip routes are then summed with the result of an operation, resulting in additional direct flow of information during forward and direct gradient flow on the backward pass.
Residual connections applied between arms of an encoder--decoder create what is called an \textit{U--Net} architecture.
In the case of super--resolution processing the forward skips can be viewed as routes for transporting unprocessed low--frequency information.
This information can be used during the decoding step in the encoder--decoder scheme.
Another way to understand residual connection is to look at them as local ensembles of shallow networks.

\subsection{Measuring quality of image--generating neural networks}

Both super--resolution networks and data augmentation networks input and output images.
Quantitive evaluation of such networks require comparison of two images---the network output and the ground truth reference image.
Images are usually compared using metrics like \textit{mean absolute error}, \textit{mean square error} and \textit{peak signal to noise ratio (PSNR)}.
These calculate error between pairs of corresponding pixels in different ways.
However these metrics may be insufficient for super--resolution related problems.
Calculating pixel--wise differences doesn't resemble the way humans estimate image quality.
Images of varying perceived quality can have same \textit{PSNRs} compared to the reference image.

To measure image similarity in more reliable way \textit{structural similarity index (SSIM)} \cite{wang-2004-ssim} was introduced.
\textit{SSIM} calculates image quality in three components:
\begin{itemize}
	\item Average \textit{luminance}.
	\item \textit{Contrast} as standard deviation of pixels.
	\item \textit{Structure} as luminance difference divided by standard deviation.
\end{itemize}
However, these values are not calculated globally.
Instead \textit{SSIM} values are measured using windows with pixel weights determined by Gaussian distribution.
Values of \textit{SSIM} components are combined using a compound formula.
Formal description of the \textit{SSIM} metric can be found in.
\textit{SSIM} has a value between zero and one, where one means a perfect match between compared images.
Having values in constrained a constrained range is another advantage of \textit{SSIM} over metrics like \textit{PSNR}.
Advantages of \textit{structural similarity index} render it suitable for super--resolution related image quality evaluation.
However, modified versions of the previously mentioned traditional metrics can also prove to be useful in the image comparison, one of them being the \textit{cPSNR}.
The traditional PSNR has a potential drawback of being sensitive to bias in image brightness.
This metric equalizes average brightness of compared images before calculating standard PSNR to alleviate this problem.
Various of the mentioned metrics were in this work, in accord to specific requirements of each step in the augmentation and super--resolution training process.

Another challange often encountered during super--resolution evaluation consists in aligning image pairs correctly.
Often two images that are to be compared are slightly shifted; it is common for these dislocations to lay in sub--pixel domain.
The process of aligning two similar images is called \textit{registration}.
Registration can be performed either with traditional or deep learning based algorithms.

\section{Super--resolution with HighRes--net}
In recent years many super--resolution architectures have emerged due to advancements in deep learning techniques.
At the moment the state--of--the art model is RAMS (\textit{Residual Attention Multi-image Super-resolution}).
However in this work \textit{HighRes--net} architecture is utilized.
\textit{HighRes--net}, which is few months older, achieves slightly worse results, however it is simpler and faster to train \cite{paperswithcode-ranking}.
Because the aim of the work is to compare different data generation techniques, not the super--resolution algorithms themselves, the more manageable architecture was chosen.
A brief description of the more sophisticated architecture is also given to provide a wider context.

\subsection{Architecture overview}
\textit{HighRes--net} \cite{deudon-2020-highresnet} is a super--resolution network based on generative deep learning.
It falls into the category of \textit{multi--frame super--resolution (MFSR)} algorithms, which takes \textit{many--to--one} (or \textit{multi--image}) approach to output generation.
In MSFR systems input is a series of images, taken with a slight shift, perhaps with a small time interval.
The input series contains more information, then a single image, as a result of random displacements, noise disturbances and atmospheric conditions.
MSFR tackles the problem of aliasing in sampled data.
Low frequency parts of image, with large geometry and little detail don't differ much between many images.
However MSFR is crucial when enhancing small detailing.
Upscaling small details from a single images can be non reliable due to aliasing.
Applying MSFR techniques and multiple low--resolution images fusion leads to de--aliasing information contained in the images.
\textit{HighRes--net} processing is divided into four subtasks:
\begin{enumerate}
	\item \textbf{Co--registration}, which estimates relative geometric differences between input images. These include divergences, due to shifts, rotations, deformations, etc.)
	\item \textbf{Fusion}, which combines multiple input images into single one, that is more refined.
	\item \textbf{Up--sampling}, which upscales low into high--resolution image.
	\item \textbf{Registration--at--the--loss}, which estimates relative geometric differences of high--resolution prediction nad ground truth, for more representative loss calculation. After calculating shift between super--resolution output and reference image, they are aligned using Lanczos resampling and then loss is measured.
        % TODO: ref shiftnet
	    The registration and alignement are learned by a model inspired by a \textit{ShiftNet} network architecture.
\end{enumerate}
The unique feature of \textit{HighRes--net} is that all of the above are learned in a single architecture in an end--to--end fashion.

\subsection{Super--resolution inference process}
The inference pipeline of HighRes--net is shown in the figure \ref{fig:highresnet-inference}.
The consecutive paragraphs will walk through each step in the process and explain how super--resolution is performed.
\begin{figure}
    \centering
    \includegraphics[width=\textwidth]{high_res_net_inference}
    \caption{Schematic of inference in \textit{HighRes--net}}
    \label{fig:highresnet-inference}
\end{figure}

The key element of \textit{HighRes--net} is achieving \textit{multi--frame super--resolution} by \textit{recursive fusion}.
Image generation is done by a neural network organized in an encoder--decoder scheme.
The input of the encoder is constructed from a series of low--resolution images.
If necessary the input set is padded with zero--valued images, to ensure that the number of low--resolution images in a power of 2, which is required by the network architecture.
For each input series a \textit{reference image} is computed using median values of images.
Then the reference picture is paired with the input images.
Each low--resolution and reference pair is processed through an embedding function.
Embedding layer consists of a convolutional layer and two residual blocks with PReLu activations.
For input of length \textit{n}, output of the encoding consists of \textit{n} images, each convolved with the reference image.
In this scheme embedding learns to perform a process called \textit{implicit co--registration}, which is responsible for adjusting geometric differences between images in the input.
It is important to notice that the embedding block is a single instance shared between input pairs.

The next step in the \textit{HighRes--net} architecture is \textit{recursive--fusion}.
In this process output images are recursively fused together, pair by pair.
Fusion operation consists of two steps---co--registration of input pair and the actual fusion.
The co--registration of fused images is similar to the co--registration of input--reference pairs.
It is done by convolutional layer with PReLu activation and two residual layers.
Then the fusion itself is done, again by a combination convolutional layer and PReLu (this part doesn't include local residual layer).
The whole co--registration--fusion includes a residual connection.
Similarly to the embedding block, the fusion operator has a single instance that is shared for all steps of the recursion.

The last step of super--resolution process is to upscale the image, by decoding the hidden state.
This is done with transposed convolutional layer with PReLu activation.
The transposition of the output of convolution makes the data grow in spatial dimensions, instead of the usual increase of depth when convolving.
The final image is constructed by applying convolution of size one, which doesn't change the size of the image.

\subsection{Registered loss calculation}
As stated before registration is an important part of \textit{HighRes--net} architecture.
It is especially crucial at loss calculation step.
Without registration the network would learn to output blurry images, as a result of shift between predictions and targets.
Previous steps of \textit{HighRes--net} include an \textit{implicit co--registration}, where registration mechanisms learned by the network don't have to be necessarily based on shifts, but also other geometric distortions.
During evaluation it is desired to register image shifts explicitly, thus the \textit{registration--at--loss} differs from the registration performed during encoding and fusion.
At the final step the sub--pixel registration is done by the \textit{ShiftNet--Lanczos} network.
\textit{ShiftNet} \cite{zhaoyi-2018-shiftnet} was introduced before \textit{HighRes--net}, in a separate research.
It was created with image inpainting via \textit{Deep Feature Rearrangement}.
Because this kind of filling missing picture areas works by reusing and transferring existing data it is suitable to be used as a registration mechanism.
It implements a modified \textit{U--Net} \cite{ronnenberger-2015-unet} architecture.
As mentioned in the introduction, U--Nets follow the encoder--decoder pattern with multiple residual connections.
Pairs of convolution and deconvolution layers in the contracting and expanding arms of a U--Net feature a residual connection.
The \textit{ShitNet} variant of \textit{U--Net} architecture contains an additional \textit{shift} operation for one of these residual connections.
More about \textit{ShiftNet} can be found in the publication.

\section{Other super--resolution architectures}
As mentioned, other super--resolution architectures are available, with RAMS \cite{salvetti-2020-rams} being the best performing one.
RAMS utilizes a novel technique called \textit{feature attention mechanism}, which enables the network to focus on high--frequency information that can be used to produce more detailed outputs.
This leads to overcoming main locality limitations of convolutional operations.
Mechanism used in RAMS are specifically aimed at multi--image super--resolution of remote sensing data.
RAMS approach takes into account the nature of satellite imagery---relatively low spatial resolution and high depth and temporal resolution.
The attention mechanism works with three dimensional convolutions to explore all possible directions.
This architecture puts emphasis on simultaneous data exploration and from spatial and temporal dimension resulting in best quality of multi--image super--resolution.

\chapter{Data augmentation with usage of deep learning}
% Thing to note: UpSampling vs ConvTranspose

\chapter{Super--resolution training and evaluation}
\section{Training HighRes--net}
As stated in the experiment layout the final step consists in training HighRes--net super--resolution model with different data.
One final step before conducting the training is to examine visually the results of augmentation on the Sentinel--2 image.
These are presented for the simple convolutional augmentation network architecture in the figure \ref{fig:export-example} and \ref{fig:export-example-zoomed}.
\begin{figure}
    \begin{subfigure}[t]{0.3\textwidth}
        \centering
        \includegraphics[width=\textwidth]{sentinel_export_hr}
        \caption{Low--resolution}
    \end{subfigure}
    \hfill
    \begin{subfigure}[t]{0.3\textwidth}
        \centering
        \includegraphics[width=\textwidth]{sentinel_export_pred}
        \caption{Low--resolution augmented}
    \end{subfigure}
    \hfill
    \begin{subfigure}[t]{0.3\textwidth}
        \centering
        \includegraphics[width=\textwidth]{sentinel_export_bicubic}
        \caption{Low resolution resized with bicubic}
    \end{subfigure}
    \caption{Example of Sentinel--2 training data export with simple convolutional augmentation network}
    \label{fig:export-example}
\end{figure}
\begin{figure}
    \begin{subfigure}[t]{0.3\textwidth}
        \centering
        \includegraphics[width=\textwidth]{sentinel_export_zoomed_hr}
        \caption{Low--resolution}
    \end{subfigure}
    \hfill
    \begin{subfigure}[t]{0.3\textwidth}
        \centering
        \includegraphics[width=\textwidth]{sentinel_export_zoomed_pred}
        \caption{Low--resolution augmented}
    \end{subfigure}
    \hfill
    \begin{subfigure}[t]{0.3\textwidth}
        \centering
        \includegraphics[width=\textwidth]{sentinel_export_zoomed_bicubic}
        \caption{Low resolution resized with bicubic}
    \end{subfigure}
    \caption{Zoomed example of Sentinel--2 training data export with simple convolutional augmentation network}
    \label{fig:export-example-zoomed}
\end{figure}

As planned, the super--resolution is to be trained with multi--image data in a single--band (eight band) mode.
Training is configured in such a way, that only weights for the best validation score are saved.
However, automatic stopping is not enabled, for this reason fitting was interrupted manually around epoch one hundred, when all training curves plateaued.
Training history has been plotted to visualize the loss fitting progress.

\subsection{Cross validation as stop condition}
Having multiple datasets generated in different ways, enables an alternative way of validating the training process.
Instead of taking a subpart of the training set as validation data, one may use images generated in a different way.
Such a technique may help to investigate the generalization capabilities of the super--resolution network.
The datasets created using the simple convolutional network and resizing algorithm were used for the cross validation.
In each of them, one set was used for fitting and the other was used for validation.
This time the increasing validation loss indicates overfitting in early epochs, compared to previous trainings.
This may indicate that the vast part of the fitting process does not contribute to overall generalization capability of the super--resolution network.
The cross--validated trainings were stopped earlier; since only weights for the epoch with best validation score are saved it is pointless to run longer fittings.
Figure \ref{fig:highres-net-training-loss} shows the loss history for all discussed HighRes--net trainings, figure \ref{fig:highres-net-training-validation} presents validation loss improvement.
\begin{figure}
    \centering
    \begin{tikzpicture}
			\begin{axis}[width=\linewidth, height=10cm, grid=major, grid style={dashed}]
			\addlegendentry{Bicubic}
%			\addplot+[mark=none] table [x=step, y=value, col sep=comma] {data/};
			\addlegendentry{Simple convolutional}
			%			\addplot+[mark=none] table [x=step, y=value, col sep=comma] {data/};
			\addlegendentry{Encoder--decoder}
			%			\addplot+[mark=none] table [x=step, y=value, col sep=comma] {data/};
			\addlegendentry{GAN}
			%			\addplot+[mark=none] table [x=step, y=value, col sep=comma] {data/};
			\addlegendentry{t: Simple convolutional, v: Bicubic}
			%			\addplot+[mark=none] table [x=step, y=value, col sep=comma] {data/};
			\addlegendentry{t: Bicubic, v: Simple convolutional}
			%			\addplot+[mark=none] table [x=step, y=value, col sep=comma] {data/};
			\end{axis}
		\end{tikzpicture}
    \caption{HighRes--net training loss history}
    \label{fig:highres-net-training-loss}
\end{figure}
\begin{figure}
    \centering
    \begin{tikzpicture}
			\begin{axis}[width=\linewidth, height=10cm, grid=major, grid style={dashed}]
			\addlegendentry{Bicubic}
%			\addplot+[mark=none] table [x=step, y=value, col sep=comma] {data/};
			\addlegendentry{Simple convolutional}
			%			\addplot+[mark=none] table [x=step, y=value, col sep=comma] {data/};
			\addlegendentry{Encoder--decoder}
			%			\addplot+[mark=none] table [x=step, y=value, col sep=comma] {data/};
			\addlegendentry{GAN}
			%			\addplot+[mark=none] table [x=step, y=value, col sep=comma] {data/};
			\addlegendentry{t: Simple convolutional, v: Bicubic}
			%			\addplot+[mark=none] table [x=step, y=value, col sep=comma] {data/};
			\addlegendentry{t: Bicubic, v: Simple convolutional}
			%			\addplot+[mark=none] table [x=step, y=value, col sep=comma] {data/};
			\end{axis}		\end{tikzpicture}
    \caption{HighRes--net validation loss history}
    \label{fig:highres-net-training-validation}
\end{figure}

\section{Evaluation and results}
According to the experiment layout the trained super--resolution models are to be evaluated on a variety of test datasets to examine generalization capabilities and robustness.
As stated before, the evaluation datasets include:
\begin{itemize}
	\item Test subsets from all augmented Sentinel--2 datasets, that can be used for calculating numerical metrics.
	\item Real--life Sentinel--2 images, that were not used in the training process. These do not include high and low--resolution pairs, so only visual examination can be performed.
	\item Proba--V test dataset, that can be used for calculating numerical metrics.
\end{itemize}
It should be noticed that the later two test sets differ in the GSD parameter from the synthetic low--resolution images in the Sentinel--2 training datasets.
This data is still valid for evaluation and investigation, because it is desirable that super--resolution, and machine learning algorithms in general, work on objects of any scale and size.
Limiting tests to single GSD would draw incomplete picture of the results.
The results of evaluation are presented in the table \ref{tab:super-res-results}, where rows designate test datasets and columns indicate models trained on data created in different ways.
For the cross--validation training scenarios letters \textit{t} and \textit{v} indicate training and validation datasets respectively.
The \textit{cPSNR} score was used as the super--resolution evaluation metric.

The best results are observed on the diagonal of the table, which means that the algorithm works best on test data created in the same way as training data for a given model.
This result is expected and indicates that no gross mistakes were made in the course of the work.
\begin{sidewaystable}
\caption{Evaluation of super--resolution training on different test sets}
\label{tab:super-res-results}
\begin{adjustbox}{center}
\small
\begin{tabular}{lcccccc}
\toprule
cPSNR &
  Bicubic &
  Simple conv &
  Encoder--decoder &
  GAN &
  \begin{tabular}[c]{@{}c@{}}t: Simple conv\\ v: Bicubic\end{tabular} &
  \begin{tabular}[c]{@{}c@{}}t: Bicubic\\ v: Simple conv\end{tabular} \\
\midrule
Bicubic                      &  35.78 & 28.96 & 24.55 & 25.80 & &  \\
Simple conv                  & 33.69 & 36.15 & 27.16 & 29.83 &  &  \\
Encoder--decoder             & 31.72 & 31.78 & 34.43 & 30.22 &  &  \\
GAN                          & 28.63 & 28.63 & 27.27 & 35.72 &  &  \\
Artifacts on real Sentinel 2 & yes & yes  & yes & yes & no & no \\
Proba V                      & 43.36 & 42.02 & 40.43 & 40.91 &  &  \\
\bottomrule
\end{tabular}
\end{adjustbox}
\end{sidewaystable}
As stated before the results on the Sentinel--2 real--life image were evaluated visually, by observing artifacts in the images.
The figure \ref{fig:sentinel-2-real-artifacts-simple-conv} shows artifacts presence on test image for networks trained on augmented data with and without cross--validation.
The artifacts take form of mosaic--like overlay or checkerboard effect distorting the image, especially in high frequency areas.
The cross--validated model that was stopped in the earlier epoch shows less visible artifacts.
The same observation can be made looking at the figure \ref{fig:sentinel-2-real-artifacts-bicubic}, which shows the same phenomenon for training set created using bicubic interpolation.
These two observations are marked in the results table \ref{tab:super-res-results}.
\begin{figure}
    \begin{subfigure}[t]{0.45\textwidth}
        \centering
        \includegraphics[width=\textwidth]{example-image}
        \caption{Trained without cross--validation}
    \end{subfigure}
    \hfill
    \begin{subfigure}[t]{0.45\textwidth}
        \centering
        \includegraphics[width=\textwidth]{example-image}
        \caption{Trained with cross--validation on data created by bicubic interpolation}
    \end{subfigure}
    \caption{Artifacts presence on Sentinel real--life data for network trained on dataset augmented with simple convolutional network}
    \label{fig:sentinel-2-real-artifacts-simple-conv}
\end{figure}
\begin{figure}
    \begin{subfigure}[t]{0.45\textwidth}
        \centering
        \includegraphics[width=\textwidth]{example-image}
        \caption{Trained without cross--validation}
    \end{subfigure}
    \hfill
    \begin{subfigure}[t]{0.45\textwidth}
        \centering
        \includegraphics[width=\textwidth]{example-image}
        \caption{Trained with cross--validation on data augmented with simple convolutional network}
    \end{subfigure}
    \caption{Artifacts presence on Sentinel real--life data for network trained on dataset created by bicubic interpolation}
    \label{fig:sentinel-2-real-artifacts-bicubic}
\end{figure}

The results in the table \ref{tab:super-res-results} can be summarized in the following statements:
\begin{itemize}
	\item Data created with augmentation techniques, both traditional and deep learning--based is suitable for training super--resolution networks.
	\item The different deep learning architectures give fairly similar results, although the simplest one works best.
	\item At the moment the bicubic interpolation gives slightly better results, possible reasons for that and suggestions of improvement are included in the summary.
	\item The cross--validation proves that all training are prone to overfitting to the data generation techniques. Early stopping based on different datasets can prevent this phenomenon and reduce artifacts on real--world data.
\end{itemize}

\chapter{Results}
\input{chapters/results}

\begin{appendices}
    \chapter{Model and training parameters of the augmentation networks}
\label{ch:appendix-params}
As stated in the work the models and training process are parametrized by set o variables in a \textit{params.yaml} file.
This file is supervised by Git and \gls{dvc} to ensure reproducibility and artifacts caching.
Contents of the configuration file during the training of augmentation networks are presented in Listing \ref{lst:params-file}.
\begin{longlisting}
\inputminted{yaml}{listings/params.yaml}
\caption{Contents of the \textit{params.yaml} file used for training}
\label{lst:params-file}
\end{longlisting}

\chapter{Technical documentation}
\label{ch:appendix-technical}
As mentioned in Section \ref{sec:exp-management} trainings are done either done with the aid of the \gls{dvc} system or by running scripts manually.
The trainings done automatically by the \gls{dvc} system contain the word \textit{dvc} as the experiment name.
The status of these trainings is supervised by \gls{dvc}, current progress can be checked with the \mintinline[breaklines]{shell}{dvc status} command.
Trainings are automatically rerun or recached based on configuration files after running the \mintinline[breaklines]{shell}{dvc repro} instruction.
Manual trainings are run by invoking Python directly.
This can be done with the command: \mintinline[breaklines]{shell}{python -m cnn_res_degrader.train [-s] [-a] [-g] training_name} where the optional arguments decide which architecture to train and the positional argument is the experiment name.

Validation on Proba-V dataset can be run with the command: \mintinline[breaklines]{shell}{python -m cnn_res_degrader.test [-s|-a|-g] weights_path output_dir} where the positional stores model architecture and positional arguments hold paths to trained model weights and output directory.

The augmented Sentinel-2 datasets can be created by running the command: \mintinline[breaklines]{shell}{python -m export_sentinel.py [-s|-a|-g] [-r] [-d] weights_path}, where architecture and path to weights are denoted as described earlier.
The last two of the switches enable generating random translations (instead of using ones saved in sentinel files) and running a demo export on one image (instead of using the whole dataset).
\end{appendices}

% Lists of objects
\listoffigures
\listoftables
\listoflistings

% Bibliography
\nocite{*}
\bibliographystyle{plplain}
\bibliography{references}

\end{document}
